{This report examines actual data gathered by Statistics Canada on the prevalence of diabetes in the four most populous provinces of Canada (Ontario, Quebec, British Columbia, and Alberta) and the as well as national data,  from 2015 to 2021. The report gives a summary of a C programming project that includes computations, the creation of graphs, and conclusions based on the gathered data.}

{The project requires the use of the C programming language in order to take data from a CSV file, do computations, and provide the required output, including tables and graphs. The data file includes information on the prevalence of diabetes among people aged 35 and older in each of the four provinces as well as across the entire country (excluding territories).}

{In-depth discussion of the project's essential elements is provided in the report, including computation of annual averages, identification of the provinces with the highest and lowest percentages of diabetics, and computation of the provincial and national averages of the population with diabetes diagnoses. The report also highlights the necessity to identify the provinces above and below the national average as well as the years with the highest and lowest percentages of diabetes.}

{The project also requires the development of two graphs: a line plot showing diabetes percentages from 2015 to 2021 and a bar graph showing the average percentages of diabetes among the three age groups for the entire country. The study emphasizes the significance of clearly labeling the axes as well as presenting each graph with a title and a legend.}

{The overall objective of the project is to use C programming and GNUPlot features to investigate the prevalence of diabetes in Canada's four most populous provinces and draw conclusions using data collected by Statistics Canada.}

{}

